

The comparative study of electoral systems has stressed the measurement of disprortionality. Breaking this measure into the system's responsiveness and partisan bias takes the inquiry one step further---but, so far, for two-party competition only. Our method widens the scope. The measurement and analysis of partisan bias in simple plurality, single-member district systems that have multi-party competition, such as Canada, India, and the present-day United Kingdom, will place the United States in comparative perspective. Adding other dimensions of institutional variance, such as runoff elections (in France), the Alternative Vote (in Australia), or low-magnitude porportional representation (such as Chile's binominal system or Ireland's Single Transferable Vote) should add further depth to the comparisons. 


