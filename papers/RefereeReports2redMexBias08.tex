% Created 2016-06-10 Fri 17:22
\documentclass{article}
\usepackage[utf8]{inputenc}
\usepackage[T1]{fontenc}
\usepackage{fixltx2e}
\usepackage{graphicx}
\usepackage{grffile}
\usepackage{longtable}
\usepackage{wrapfig}
\usepackage{rotating}
\usepackage[normalem]{ulem}
\usepackage{amsmath}
\usepackage{textcomp}
\usepackage{amssymb}
\usepackage{capt-of}
\usepackage{hyperref}
\author{Eric Magar}
\date{\today}
\title{Elements for RnR's cover letter}
\hypersetup{
 pdfauthor={Eric Magar},
 pdftitle={Elements for RnR's cover letter},
 pdfkeywords={},
 pdfsubject={},
 pdfcreator={Emacs 24.3.1 (Org mode 8.3.4)}, 
 pdflang={English}}
\begin{document}

\maketitle
\tableofcontents


\section{{\bfseries\sffamily NO\_ACTION\_NEEDED} Editor's letter}
\label{sec:orgheadline1}
Ref.:  Ms. No. PG-2218

Dear Dr. Eric Magar,

Three expert reviewers have now commented on your manuscript. Thank you for waiting. Based on these reviews and my own reading, I invite you to resubmit a revised manuscript. You will see from the review reports appended below that all reviewers see considerable merit in your paper, although they also offer constructive advice on how to make the contribution clearer.

Please let me know within the next couple of weeks whether you will proceed to revise and resubmit this manuscript for Political Geography. The revision should be accompanied with a brief anonymous response letter to the referees. Explain what you did to meet their feedback (or did not do, and why). This response must be anonymous (that is, do not sign the letter).

We need the revision back within the next three months. It will then go back to the same referees for another look. The resubmission must stay below 11,000 words (inclusive) to be acceptable for further processing.

To submit a revision, go to \url{http://ees.elsevier.com/jpgq/} and log in as an Author.  You will find your submission record under the menu item 'Submission Needing Revision'.

Your username is: emagar@itam.mx

If you need to retrieve password details, please go to:
\url{http://ees.elsevier.com/jpgq/automail_query.asp}

PLEASE NOTE: The journal would like to enrich online articles by visualising and providing geographical details described in Political Geography articles. For this purpose, corresponding KML (GoogleMaps) files can be uploaded in our online submission system. Submitted KML files will be published with your online article on ScienceDirect. Elsevier will generate maps from the KML files and include them in the online article.

Political Geography features the Interactive Map Viewer, \url{http://www.elsevier.com/googlemaps}. Interactive Maps visualize geospatial data provided by the author in a GoogleMap. To include one with your article, please submit a .kml or .kmz file and test it online at \url{http://elsevier-apps.sciverse.com/GoogleMaps/verification} before uploading it with your submission.

Please let me know if you have any questions or concerns.

Yours sincerely

Halvard Buhaug, PhD
Associate Editor
Political Geography





Additional Comments:

\section{{\bfseries\sffamily DONE} Response to editor accepting to do Rnr}
\label{sec:orgheadline2}
Dear Dr. Buhaug, 
It is with great pleasure that I read the good news about our submission. The reviews are constructive, offering substantive advise, and arrived rather fast! I am sure that they will help us improve the manuscript in the hope that it is acceptable for publication in Political Geography. My co-authors and I will gladly proceed with the revise and resubmit. We will send you a revised manuscript within two months. 
Best,

\section{Reviewer \#2}
\label{sec:orgheadline16}
\begin{enumerate}
\item {\bfseries\sffamily NO\_ACTION\_NEEDED} This MS deals with partisan bias, in terms of discrepancies between seat and vote shares, looking more closely at three particular sources of such bias. Using data from recent Mexican elections, the paper details a procedure on how to calculate the different sources of this bias in single-member district systems where more than two parties compete.
\label{sec:orgheadline3}
\begin{itemize}
\item EM action: no action needed, no critique.
\end{itemize}

\item {\bfseries\sffamily NO\_ACTION\_NEEDED} I have to admit from the outset that I'm no expert on this particular topic, although I am familiar with it and work on topics that are not too far removed. Hence, my comments are those of a "generalist reviewer", and are more related to the framing of the paper and different conceptual issues that are not entirely clear to me.
\label{sec:orgheadline4}
\begin{itemize}
\item EM action: no action needed, no critique.
\end{itemize}

\item {\bfseries\sffamily NO\_ACTION\_NEEDED} But, my overall impression of this MS is that it has the potential to make a nice contribution to the literature and study of partisan bias, not the least because the authors provide a very clear template describing how others could go about in calculating partisan bias in other single-member district system. The summary of sources of partisan biases and the comprehensive treatment and discussion of the issue also makes for a good overview, also for those not working closely on the topic (such as myself). The methods and design are transparent, and mostly well justified, and from what I can understand the analysis is well conducted. I think this could become a publishable article, even in a top outlet such as Political Geography, without too much extra effort on the part of the authors. Still, there are a few things that need to be revisited before the MS is publishable.
\label{sec:orgheadline5}
\begin{itemize}
\item EM action: no action needed, no critique.
\end{itemize}

\item {\bfseries\sffamily DONE} The first issue relates to the "dual framing" of the paper: There is the methodological procedure for calculating sources of bias when there are more than two parties, and there are the substantive results on party biases for the larger Mexican parties. I think the main contribution is the former and this is also mostly how this paper is framed. However, there are parts of the paper that reads as if the results for Mexican bias is the key contribution, and the "dual framing" is also reflected in the rather lengthy (but well written) sections on Mexico starting on p.13.
\label{sec:orgheadline6}
\begin{itemize}
\item Point addressed below.
\end{itemize}
\item {\bfseries\sffamily DONE} I would go for a purer methodological framing of this paper, and make even clearer that Mexico is "just" an application/illustration (even if it is an interesting one, and for a large country). This could potentially go together with shortening the discussion on Mexico, and  freeing up some more space for discussion about the methodological choices and potential problems, conceptual discussions of the various sources of bias etc. There is another particular reason why I think this way of framing the paper is advantageous: As the authors remark, Mexico is a mixed-member electoral system, with a PR tier to reduce disproportionality. This is completely left out of the authors' analysis. While the authors make some pertinent remarks about effects of partisan bias in the single-member tier on political behavior on pp.13-14, it is still the case that we cannot learn much about the overall extent of partisan bias and issues of representation in the Mexican system from this analysis: There will be in all likelihood be a huge upwards bias, and statements such as those made in the abstract concerning the "partisan bias in favor of Mexico's former hegemonic ruling party" etc will be misleading to those who just skim the paper and do not read carefully. If the authors want to make a substantive contribution on partisan bias in a particular country, they would then either need to incorporate the PR tier when discussing overall partisan bias in Mexico, or choose another country  that is a pure single-member system.
\label{sec:orgheadline7}
\begin{itemize}
\item EM action: conceded, took framing from earlier version that actually emphasized methodological contribution; downplayed the Mexican case.
\item Point addressed below.
\item Might raise issue with other reviewers?
\end{itemize}
\item {\bfseries\sffamily DONE} However, if the authors rather chose to frame this even more clearly as a methodological contribution with single-member tier of Mexico as an illustration, I think this issue is not too big (substantive conclusions on overall bias in Mexico would still need to come with some clear caveats, however, so the abstract, for example, would need re-phrasing.
\label{sec:orgheadline8}
\begin{itemize}
\item EM action: conceded.
\item Reviewer \#2 recommended reframing the manuscript more clearly as a methodological contribution with analysis of the single-member tier of Mexico as an illustration. We have adopted this recommendation, rewriting the introduction and the abstract accordingly. We also trimmed the disussion of Mexico considerably, especially section 4 on Diputado elections, but also section 5 on malapportionment. This is an improvement in the manuscript, with sharp focus in the method to measure partisan bias sources in multi-party competition. We acknowledge the referee's suggestion, as it helped avoid the thorny issue of dropping the PR tier frm the analysis.
\end{itemize}
\item {\bfseries\sffamily NO\_ACTION\_NEEDED} I basically buy the approach, and the different sources of bias all seem plausible and are well discussed. Still, there are a couple of things that could be discussed/elaborated on:
\label{sec:orgheadline9}
\begin{itemize}
\item EM action: no action needed, no critique.
\end{itemize}

\item {\bfseries\sffamily DONE} First, it is not clear from the outset how measuring pure partisan bias is/can be differentiated from biases related to the size of the party and characteristics of the system favoring larger parties in general (not because of the party's identity, but because of its size). For example, consider two hypothetical elections in which two parties A and B receive the exact same vote share in all districts in a given election. In election 1, A receives 51\% and B 49\% of votes, in all districts, and A receives all delegates. In election 2, B receives 51\% in all districts and all delegates. Now, to me, I wouldn't say that the system is necessarily biased in favor of  A in the first election and B in the second; this is not about the party's identity but about the responsiveness of the system (which does not change from election 1 to 2). An early clarifying discussion, which should be really simple so that non-experts can understand, of how such issues are conceptually handled when delineating what is partisan bias, and a clarification if this truly matters for any of the sources of bias calculated would be very welcome.
\label{sec:orgheadline10}
\begin{itemize}
\item Conceded. As suggested by Reviewer 2, we have expanded the discussion of the rho parameter in section 1 to attempt a clarification of how measuring pure partisan bias is/can be differentiated from biases related to the size of the party and characteristics of the system favoring larger parties in general (not because of the party's identity, but because of its size).
\end{itemize}

\item {\bfseries\sffamily HALFWAY} Second, given the definition of party bias, I see why the "turnout-based" bias should be part of it. However, one problem here that could be discussed is that turnout in a given district is not something that is exogenously given, but rather affected by the actions and campaign strategies of the different parties. Parties may, for example, campaign harder to get out the vote in districts that are close to call, and some parties may simply be better at winning such districts. Hence, if, let's say the PRI wins a lot of high-turnout districts, it could be due to some inherent actions and capacities of the party, and in a sense it would be wrong to consider this a bias against the PRI in the system. I don't think this is a very big issue, but I think it merits some discussion.
\label{sec:orgheadline11}
\begin{itemize}
\item EM action: concede, add short discussion in text \textbf{where?}
\item In line with Federico's critique\ldots{} should talk to him.
\item Of the three components, malapportionment is the easiest to assess. Turnout has an endogenous nature (Cox ans someone). Border delimination may be attempt at gerrymandering, but could simply be an accident of geography. Malapportionment is associated with the census gap.
\item Even if two components are harder to interpret, the fact that the method separates malapportionment from the rest makes it an interesting contribution.
\item Further theory on factors driving the other two components will make their interpretation easier.
\item We are grateful to Federico Estévez and one anonymous referee for pointing this out.
\end{itemize}

\item {\bfseries\sffamily DONE} Considering PRI-Green as an alliance and assigning all wins to PRI seems to be a major assumption (which the authors are open about). Now, even if the substantive results for Mexico is downplayed, it would also be interesting from a general point of view to know by how much such assumptions alter the results and conclusions. What happens to the results if PRI-Green is measured as one entity, for example (as suggested on top p.16)?
\label{sec:orgheadline12}
\begin{itemize}
\item EM action: conceded.
\item We have kept the manipulation described in the original manuscript in the text. We have also added a section in the on-line appendix devoted to elaborating two other approaches to handle partial coalitions: (1) one where the Green is summed to the PRI across the board, (2) another where the PRI-Green is treated as one entity. We re-estimated the 2015 election using these approaches, and compared them to reported results. We conclude that results change in predictable ways when coalitions are handled differently. We also underscore that partial coalitions are a Mexico specific feature that should pose no obstacle to estimation in other multi-party compatitions cases.
\end{itemize}

\item {\bfseries\sffamily CONCEDE} The rationale for studying partisan bias for each election separately is well explained. However, what if one wants to generalize and test for a systematic bias (or even particular systematic such sources) for/against a party within a system that lasted for a specific period of time? I understand that creeping malapportionment and turnout will change by the election, but if researchers want to make such generalizations, how could they apply/alter your framework to produce (at least rough) tests of this. It's interesting to come up with exact numbers for a particular election, but sometimes the question is whether this is an inherent/more systematic feature of the system, and as the authors show some sources of bias show substantial changes between elections, favoring a party in one and disfavoring it in the next. If the authors could devise a strategy for conducting such a test, this would greatly enhance their contribution, I think.
\label{sec:orgheadline13}
\begin{itemize}
\item No clue on how to articulate this\ldots{} longitudinal pool of Linzer sims? longitudinal a la Marquez? State-level multiplications? Something else?
\end{itemize}

\item {\bfseries\sffamily DONE} Very minor point: Population levels in districts are based on linear interpolations between censuses. Following standard models of population growth, it would be more appropriate to assume constant population growth rates over the time interval, which can easily be calculated (meaning that population growth in absolute numbers will be smaller for earlier years, if growth is positive).
\label{sec:orgheadline14}
\begin{itemize}
\item EM action: defend our approach, discuss alternative in on-line appendix.
\item Estimating intercensal populations in units of analysis is no small task. The key problem appears to be the choice of a functional form that both smoothes the rate of population growth while also taking the values actually observed on three census years (2000, 2005, and 2010). An exponential form between pairs of census does a good job for years between observations, but not before and after, nor does it treat "transitions from one pair to the next smoothly. A polinomial form would allow work with all three census counts, but also seems problematic for proecting estimates beyond 2010. Since all this requires demographic knowledge beyond our ability, we opted for the simpler linear estimation instead. We elaborate our linear estimation method, and the challenges of the non-linear approach, in the on-line appendix.
\end{itemize}
\item {\bfseries\sffamily NO\_ACTION\_NEEDED} All in all, I learned a lot from reading this paper, and I think it is a very good piece of scholarly work. I recommend that the authors are given the chance to revise and resubmit the paper.
\label{sec:orgheadline15}
\begin{itemize}
\item EM action: no action needed, no critique.
\end{itemize}
\end{enumerate}

\section{Reviewer \#3}
\label{sec:orgheadline20}
\begin{enumerate}
\item {\bfseries\sffamily NO\_ACTION\_NEEDED} This article was a pleasure to read and to evaluate. It is well written, with a theory clearly presented, interesting findings, and a contribution to the estimation of seats and votes in multi-party systems. The article brings together three different traditions in the study of seats and votes, unifying them in a model that allows researchers to discriminate the sources of biases in multi-party races. The article will be of interest to those that conduct basic research on seat-vote models as well as those interested in the mechanical properties of electoral rules in Mexico. Consequently, I recommend publication as is.
\label{sec:orgheadline17}
\begin{itemize}
\item EM action: no action needed, no critique.
\end{itemize}

\item {\bfseries\sffamily NO\_ACTION\_NEEDED} I do not have recommended changes to the article. The article is honest in stating that it is an original and interesting improvement on existing models but not a radically different modeling strategy. I appreciate that the article does not try to oversell their contribution or findings.
\label{sec:orgheadline18}
\begin{itemize}
\item EM action: no action needed, no critique.
\end{itemize}

\item {\bfseries\sffamily CONCEDE} That said, the authors could be more aggressive in the introduction to convey to readers how the current article changes prior conventional wisdom in the Mexican election and what contributions will result from estimating their model in other electoral systems. Beyond that, I could only hope that every manuscript I have to review would be such an easy and interesting reading.
\label{sec:orgheadline19}
\begin{itemize}
\item EM action: work in progress
\end{itemize}
\end{enumerate}

\section{Reviewer \#4}
\label{sec:orgheadline31}
\begin{enumerate}
\item {\bfseries\sffamily NO\_ACTION\_NEEDED} This paper intends to identify the relative and (assumed independent) impact of three different components of partisan bias  in the Mexican electoral system. Application case are lower-chamber federal legislative elections 2003-2012. The paper focuses exclusively on the single member districts  component of the Mexican electoral system.
\label{sec:orgheadline21}
\begin{itemize}
\item EM action: no action needed, no critique.
\end{itemize}

\item {\bfseries\sffamily DONE} The objective of this paper is ambitious. The authors claim to combine most important methodical contributions in this context (Grofman et al. 1997, King 1990, Linzer 2012).  More specifically, it aims at separating the relative impact of malapportionment, boundary delimitations, and differential turnout in an additive multinomial logit model. Sometimes, one has the impression, that a failed redistricting reform is at the center of the paper --- which is confusing.
\label{sec:orgheadline22}
\begin{itemize}
\item EM action: conceded.
\item By de-emphasizing the Mexican case in order to highlight the methodological contribution, we have also done our best to remove Reviewer 4's impression that the failed redistricting reform is confusingly at the center of the paper. Analysis uses the map that was not implemented as part of the hypothetical analysis, offering perspective on the effect of reducing malapportionment (much else constant) on our measure of partisan bias and its components.
\end{itemize}
\item {\bfseries\sffamily DONE} Description and discussion of the Mexican electoral system is negligent. The author writes: "Section 4 describes Mexico's mixed-member electoral system, isolating the plurality tier for analysis" (p. 5). The description of the electoral system can actually be found in footnote 5. But  what does 'isolating' mean? The authors state: "We examine, in isolation, the elections held in the single member plurality-win districts. We do so because all voting and most campaigning take place in the plurality tier." The reviewer considers this legitimation as not sufficient. Note that the current Mexican electoral system includes stipulations balancing excessive partisan bias and including compensation schemes. More specifically, the electoral law prescribes an upper bound of seat-vote deviation of 8\%. The is not referred in the paper. Why ? Actually, estimating partisan bias for the SMD component exclusively without referring to this stipulation seems to be misleading.
\label{sec:orgheadline23}
\begin{itemize}
\item EM action: conceded.
\item While analysis of SMD seats without the compensatory PR tier can be defended, it is no longer necessary. By reframing the paper as a methodological contribution with an illustrative (and interesting) application to the plurality tier of the Mexican electoral system---as advised by Reviewer 2---it is justified to ignore the PR tier. We have nonetheless added a paragraph (in section 4) elaborating how the substantive partisan bias results presented for \textbf{plurality seats only} have implications for the larger mixed system. We also added the 8 percent over-representation rule, which we had previously neglected to mention.
\end{itemize}

\item {\bfseries\sffamily DONE} One would never seriously propose, to measure partisan bias exclusively for the SMD component in a mixed system like, e.g. the German electoral system.
\label{sec:orgheadline24}
\begin{itemize}
\item EM action: conceded.
\item Answered above.
\end{itemize}
\item {\bfseries\sffamily DONE} It is reasonable to expect strategic coordination of parties, candidates and voters in this context. The paper does not propose a theory what partisan bias means in such a setting. The authors cite Calvo/Micozzi (2005) but do not systematically take into account their arguments, especially the insight that "with more than two parties the relative change in seats depends critically on changes in the number of parties" (Calvo/Micozzi p. 1051)
\label{sec:orgheadline25}
\begin{itemize}
\item EM action: defend our approach, mild concession.
\item If the vote threshold to win another seat can be anticipated, strategic coordination is the attempt to pool votes (or remove opportunities to spread votes thin) in order to reach that threshold. Other things constant, the threshold should be lower for bias-favored parties than for other parties, and partisan bias should therefore remove incentives for bias-favored parties to coordinate strategically with others. If that were so, however, bias-unfavored parties have incentives to coordinate, joining forces in an attempt to overcome their disadvantage by accruing the large-party bonus associated with parameter rho. These simple statements suggest one obstacle to a theory of what partisan bias means under strategic coordination: it is unlikely that other things remain constant. In particular, gauging the relation between partisan bias and strategic coordination is contingent also on parameter rho. A formal exposition of this is Cox and Katz's (2002, chapter 3) model of the value to parties of redistricting plans, where utility is a function of both lambda and rho. Since our paper is focused in lambdas only, we do not undertake such a theory.
\item Regarding multipartism, Calvo and Micozzi show that increasing party competition pulls the votes-seats curve leftward (fig. 1-c). This is precisely what partisan bias achieves---but for a favored party only: a more efficient votes to seats conversion. Bias produces a rightward shift for unfavored parties: a less efficient votes to seats conversion. So whereas the effect of multipartism remains symmetric, partisan bias does not. We have added a footnote at the end of section 1 to cover this.
\item Questions for co-authors: Should we keep the new foonote, or drop this? Can it be clarified?
\end{itemize}

\item {\bfseries\sffamily DONE} Section 3 is titled 'Measurement via Monte Carlo simulation'. I guess, the authors mean 'estimation'. The one-page section is not very instructive --- it provides no detailed insights to the applied procedure. The description of the modeling approach is not precise enough. (Some details are provided in footnote 14 in the results section).  More details of the formal setup (including the electoral system) are necessary. The derivation from, and the combination of the existing approaches should be precise and transparent.
\label{sec:orgheadline26}
\begin{itemize}
\item EM action: conceded.
\item Section 3 now describes the modeling approach more explicitely. We have also written an on-line appendix (included along this re-submission) to accompany the article if it gets published. The appendix elaborates on the full applied procedure in a step-by-step approach, also serving as introduction to the code to replicate the analysis (that will be posted on-line upon publication). In particular, the appendix offer detail about the Monte Carlo to generate a large number of hypothetical national elections for each year (the Linzer method), and how three methods combine into our proposed procedure.
\item We also replaced `measurement' with `estimation' in the section name.
\item The attached appendix is still a work in progress, we plan to polish it if the article gets accepted.
\end{itemize}

\item {\bfseries\sffamily TRUE?} The authors use a multinomial logit type of model --- I missed a discussion of the crucial assumption of the independence of irrelevant alternatives (IIA) which implies equal substitution patterns which may not be met.
\label{sec:orgheadline27}
\begin{itemize}
\item work in progress
\item see King p. 168 for 1st point, true?
\end{itemize}
\item {\bfseries\sffamily HALFWAY} How do the authors account for districts with varying sets and sizes of candidates in the estimation?
\label{sec:orgheadline28}
\begin{itemize}
\item EM action: clarified in text and expanded in the on-line appendix.
\item Expand on Linzer's patterns of competition and my dummies (two ways: adapt code for exact number of parties in election, or use dummies to drop redundant columns). Expanded in appendix, with a mention in text.
\end{itemize}
\item {\bfseries\sffamily DEFEND} On p. 21, the authors state: "Leaving aside the question of how meaningful the estimated quantities are\ldots{}"  I admit that this statement is somewhat disturbing. The interpretation and usage of the estimated effects seem to be problematic: it is meanwhile established knowledge that coefficients of multiple nonlinear functions (as in MNL)  cannot be interpreted simply based on statistical significance, and even on the sign of a coefficient. Covariates have to be explicitly fixed for explicit values in order to get conditional probabilities / market shares, marginals and elasticies. Insafar the study should reassess the impact of the estimated coefficients for relevant and typical situations, and for the respective partisan biases in terms of  changes in market shares. E.g., illustrative scenarios in Table 1 could be provided for estimated coefficients.
\label{sec:orgheadline29}
\begin{itemize}
\item work in progress
\end{itemize}

\item {\bfseries\sffamily CONCEDE} Table 3 uses OLS regression for deriving swing ratios: "We derive swing ratios by regressing a party's seat shares in simulated elections on the party's simulated vote shares." Does this regression account for the uncertainity/credibility of simulated shares?
\label{sec:orgheadline30}
\begin{itemize}
\item work in progress
\end{itemize}
\end{enumerate}

\section{ToDo list}
\label{sec:orgheadline41}
\begin{enumerate}
\item Write new cover letter explaining changes. Mention that we re-did all analysis to include 2015 election returns (previously unavailable), and also adding back secciones that were split in the period due to overpopulation. These had been dropped to save time. These units are relatively unimportant in sheer numbers (175 overpopulated secciones were split into 5034 new units in the period). But they are concentrated in suburban areas with fast demographic growth since the 1990s. Estimates for 2003--2012 have changed, but they tell the same general story.
\label{sec:orgheadline32}
\item Ask Micah/Mike: Which repository for data, code, appendix? github? ericmagar.com? dataverse? several?
\label{sec:orgheadline33}
\item {\bfseries\sffamily DONE} Re-do rri plots with cleaner seccion-to-dostroct aggregations for paper
\label{sec:orgheadline34}
\item {\bfseries\sffamily DONE} Re-do bias estimate plots with 2015 in for paper
\label{sec:orgheadline35}
\item Decide if we call it the 2013 map or the 2015 map
\label{sec:orgheadline36}
\item Make sure census gap mentioned in the text: I mention it in the appendix without introduction
\label{sec:orgheadline37}
\item Mike: The two comments I received from MPSA were:
\label{sec:orgheadline40}
\begin{enumerate}
\item {\bfseries\sffamily DONE} Need a little more detail on the MCMC algorithm
\label{sec:orgheadline38}
\item Need more context for non-Mexico scholars
\label{sec:orgheadline39}
\end{enumerate}
\end{enumerate}

\section{ToDo list if we get publication}
\label{sec:orgheadline46}
\begin{enumerate}
\item Remove circularities btw red.r and analizaEscenarios.r
\label{sec:orgheadline42}
\item {\bfseries\sffamily DONE} verify that error in king's denominator in red.r is innocuous
\label{sec:orgheadline43}
\item Drop above from spaghetti code (never used for Linzer estimation)
\label{sec:orgheadline44}

\item Turn various code files (red.r, linzerElas.r, analizaEscenarios.r\ldots{}) into single---if longer---script
\label{sec:orgheadline45}
\end{enumerate}
\end{document}
