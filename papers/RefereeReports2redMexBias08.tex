% Created 2016-06-17 Fri 22:18
\documentclass{article}
\usepackage[utf8]{inputenc}
\usepackage[T1]{fontenc}
\usepackage{fixltx2e}
\usepackage{graphicx}
\usepackage{grffile}
\usepackage{longtable}
\usepackage{wrapfig}
\usepackage{rotating}
\usepackage[normalem]{ulem}
\usepackage{amsmath}
\usepackage{textcomp}
\usepackage{amssymb}
\usepackage{capt-of}
\usepackage{hyperref}
\author{Eric Magar}
\date{\today}
\title{Elements for RnR's cover letter}
\hypersetup{
 pdfauthor={Eric Magar},
 pdftitle={Elements for RnR's cover letter},
 pdfkeywords={},
 pdfsubject={},
 pdfcreator={Emacs 24.3.1 (Org mode 8.3.4)}, 
 pdflang={English}}
\begin{document}

\maketitle
\tableofcontents


\section{Cover letter for RnR}
\label{sec:orgheadline1}
Halvard Buhaug, PhD
Associate Editor
Political Geography

Dear Dr. Buhaug: My co-authors and I have revised our manuscript and are re-submitting it for your consideration and for a second review. 

We addressed all concerns by the reviewers, either in the text or through the addition of the accompanying on-line appendix. This letter explains our responses. 

We accepted all but three points raised by the reviewers -- and have corrected or clarifyied the text and analysis or elaborated based on critiques and recommendations. We quote below the points raised by the reviewers requiring our attention, following each with what we did and where, or did not do and why. 

In addition to reviewers' feedback, we updated all analysis to include 2015 election returns (data that was still unavailable when we prepared the original manuscript) and included secciones that were split in the period of observation due to overpopulation. These secciones had been dropped from the original analysis to save time (recovering them required a good deal of effort). These units are relatively unimportant in sheer numbers (175 overpopulated secciones were split into 5034 new units in the period, out of a total of 66 thousand). But they are concentrated in suburban areas with fast demographic growth since the 1990s. The revised estimates support the the same substantive conclusions, although some individual estimates have changed. 

The new on-line appendix provides detail of our estimation procedure, with a step-by-step explanation of how to prepare data, invoke hypothetical election generation, and specify the Bugs model. At time of publication, we will archive replication code, and data along with this appendix -- which will support straightforward replication.

We are confident that the review process has allowed us to improve our manuscript, and hope that the revised version will be acceptable for publication.

Yours sincerely,

Eric Magar

\section{{\bfseries\sffamily DONE} Reviewer \#2}
\label{sec:orgheadline15}
\begin{enumerate}
\item {\bfseries\sffamily NO\_ACTION\_NEEDED} This MS deals with partisan bias, in terms of discrepancies between seat and vote shares, looking more closely at three particular sources of such bias. Using data from recent Mexican elections, the paper details a procedure on how to calculate the different sources of this bias in single-member district systems where more than two parties compete.
\label{sec:orgheadline2}
\begin{itemize}
\item EM action: none.
\end{itemize}
\item {\bfseries\sffamily NO\_ACTION\_NEEDED} I have to admit from the outset that I'm no expert on this particular topic, although I am familiar with it and work on topics that are not too far removed. Hence, my comments are those of a "generalist reviewer", and are more related to the framing of the paper and different conceptual issues that are not entirely clear to me.
\label{sec:orgheadline3}
\begin{itemize}
\item EM action: none.
\end{itemize}
\item {\bfseries\sffamily NO\_ACTION\_NEEDED} But, my overall impression of this MS is that it has the potential to make a nice contribution to the literature and study of partisan bias, not the least because the authors provide a very clear template describing how others could go about in calculating partisan bias in other single-member district system. The summary of sources of partisan biases and the comprehensive treatment and discussion of the issue also makes for a good overview, also for those not working closely on the topic (such as myself). The methods and design are transparent, and mostly well justified, and from what I can understand the analysis is well conducted. I think this could become a publishable article, even in a top outlet such as Political Geography, without too much extra effort on the part of the authors. Still, there are a few things that need to be revisited before the MS is publishable.
\label{sec:orgheadline4}
\begin{itemize}
\item EM action: none.
\end{itemize}
\item {\bfseries\sffamily DONE} The first issue relates to the "dual framing" of the paper: There is the methodological procedure for calculating sources of bias when there are more than two parties, and there are the substantive results on party biases for the larger Mexican parties. I think the main contribution is the former and this is also mostly how this paper is framed. However, there are parts of the paper that reads as if the results for Mexican bias is the key contribution, and the "dual framing" is also reflected in the rather lengthy (but well written) sections on Mexico starting on p.13.
\label{sec:orgheadline5}
\begin{itemize}
\item This issue is addressed together with another below.
\end{itemize}
\item {\bfseries\sffamily DONE} I would go for a purer methodological framing of this paper, and make even clearer that Mexico is "just" an application/illustration (even if it is an interesting one, and for a large country). This could potentially go together with shortening the discussion on Mexico, and  freeing up some more space for discussion about the methodological choices and potential problems, conceptual discussions of the various sources of bias etc. There is another particular reason why I think this way of framing the paper is advantageous: As the authors remark, Mexico is a mixed-member electoral system, with a PR tier to reduce disproportionality. This is completely left out of the authors' analysis. While the authors make some pertinent remarks about effects of partisan bias in the single-member tier on political behavior on pp.13-14, it is still the case that we cannot learn much about the overall extent of partisan bias and issues of representation in the Mexican system from this analysis: There will be in all likelihood be a huge upwards bias, and statements such as those made in the abstract concerning the "partisan bias in favor of Mexico's former hegemonic ruling party" etc will be misleading to those who just skim the paper and do not read carefully. If the authors want to make a substantive contribution on partisan bias in a particular country, they would then either need to incorporate the PR tier when discussing overall partisan bias in Mexico, or choose another country  that is a pure single-member system.
\label{sec:orgheadline6}
\begin{itemize}
\item EM action: conceded, took framing from earlier version that actually emphasized methodological contribution; downplayed the Mexican case.
\item Issue addressed together with another below.
\item Might raise issue with other reviewers?
\end{itemize}
\item {\bfseries\sffamily DONE} However, if the authors rather chose to frame this even more clearly as a methodological contribution with single-member tier of Mexico as an illustration, I think this issue is not too big (substantive conclusions on overall bias in Mexico would still need to come with some clear caveats, however, so the abstract, for example, would need re-phrasing.
\label{sec:orgheadline7}
\begin{itemize}
\item EM action: conceded.
\item Elements for our response: Reviewer \#2 recommended reframing the manuscript more clearly as a methodological contribution with analysis of the single-member tier of Mexico as an illustration. We have adopted this recommendation, rewriting the introduction and the abstract accordingly. We also trimmed the discussion of Mexico considerably, especially section 4 on Diputado elections, and section 5 on malapportionment. This is an improvement in the manuscript, with sharp focus in the method to measure partisan bias sources in multi-party competition. We acknowledge the referee's suggestion, as it helped avoid the thorny issue of dropping the PR tier from the analysis.
\end{itemize}
\item {\bfseries\sffamily NO\_ACTION\_NEEDED} I basically buy the approach, and the different sources of bias all seem plausible and are well discussed. Still, there are a couple of things that could be discussed/elaborated on:
\label{sec:orgheadline8}
\begin{itemize}
\item EM action: none.
\end{itemize}
\item {\bfseries\sffamily DONE} First, it is not clear from the outset how measuring pure partisan bias is/can be differentiated from biases related to the size of the party and characteristics of the system favoring larger parties in general (not because of the party's identity, but because of its size). For example, consider two hypothetical elections in which two parties A and B receive the exact same vote share in all districts in a given election. In election 1, A receives 51\% and B 49\% of votes, in all districts, and A receives all delegates. In election 2, B receives 51\% in all districts and all delegates. Now, to me, I wouldn't say that the system is necessarily biased in favor of  A in the first election and B in the second; this is not about the party's identity but about the responsiveness of the system (which does not change from election 1 to 2). An early clarifying discussion, which should be really simple so that non-experts can understand, of how such issues are conceptually handled when delineating what is partisan bias, and a clarification if this truly matters for any of the sources of bias calculated would be very welcome.
\label{sec:orgheadline9}
\begin{itemize}
\item EM action: conceded.
\item Elements for our response: As suggested by Reviewer \#2, we have expanded the discussion of the rho parameter in section 1 to further clarify  how measuring pure partisan bias is/can be differentiated from biases related to the size of the party and characteristics of the system favoring larger parties in general (not because of the party's identity, but because of its size).
\end{itemize}
\item {\bfseries\sffamily DONE} Second, given the definition of party bias, I see why the "turnout-based" bias should be part of it. However, one problem here that could be discussed is that turnout in a given district is not something that is exogenously given, but rather affected by the actions and campaign strategies of the different parties. Parties may, for example, campaign harder to get out the vote in districts that are close to call, and some parties may simply be better at winning such districts. Hence, if, let's say the PRI wins a lot of high-turnout districts, it could be due to some inherent actions and capacities of the party, and in a sense it would be wrong to consider this a bias against the PRI in the system. I don't think this is a very big issue, but I think it merits some discussion.
\label{sec:orgheadline10}
\begin{itemize}
\item EM action: conceded, added paragraph in results section discussing volatility and turnout's endogeneity
\item Element for response: Of the three components, the mechanism of malapportionment is easiest to assess because its origin lies squarely in institutions and choices by a small group of professional actors. Both turnout and border delimitation involve effect that are conditioned on voters behavior as well. The effect of legislative choices on turnout is more challenging to estimate because mobilization has an endogenous component (cox.munger.1989, rosenstone.hansen.1993). Similarly, some effects of border delimitation spring from voters migration behavior (Johnston's similar people live nearby argument), and not directly from the legislator's intent. 

We have added a discussion in the results section that deals with this issue (tangentially, at least) in the context of partisan bias volatility in the results.
\item Comment for co-authors: pls evaluate if the new paragraph is justified, and if the tangential treatment is enough to address the referee's concern. Should we expand the discussion in the paragraph, take a different approach to address the concern, or do nothing more?
\end{itemize}
\item {\bfseries\sffamily DONE} Considering PRI-Green as an alliance and assigning all wins to PRI seems to be a major assumption (which the authors are open about). Now, even if the substantive results for Mexico is downplayed, it would also be interesting from a general point of view to know by how much such assumptions alter the results and conclusions. What happens to the results if PRI-Green is measured as one entity, for example (as suggested on top p.16)?
\label{sec:orgheadline11}
\begin{itemize}
\item EM action: retained specification showing how alternatives affect the results.
\item Elements for our response: We have kept the manipulation described in the original manuscript in the text. We have also added a section in the on-line appendix devoted to elaborating two other approaches to handle partial coalitions: (1) one where the Green is summed to the PRI across the board, (2) another where the PRI-Green is treated as one entity. We re-estimated the 2015 election using these approaches, and compared them to reported results. We conclude that results change in predictable ways when coalitions are handled differently. We also underscore that partial coalitions are a Mexico specific feature that should pose no obstacle to estimation in other multi-party competitions cases.
\end{itemize}
\item {\bfseries\sffamily DONE} The rationale for studying partisan bias for each election separately is well explained. However, what if one wants to generalize and test for a systematic bias (or even particular systematic such sources) for/against a party within a system that lasted for a specific period of time? I understand that creeping malapportionment and turnout will change by the election, but if researchers want to make such generalizations, how could they apply/alter your framework to produce (at least rough) tests of this. It's interesting to come up with exact numbers for a particular election, but sometimes the question is whether this is an inherent/more systematic feature of the system, and as the authors show some sources of bias show substantial changes between elections, favoring a party in one and disfavoring it in the next. If the authors could devise a strategy for conducting such a test, this would greatly enhance their contribution, I think.
\label{sec:orgheadline12}
\begin{itemize}
\item EM action: conceded, added paragraph.
\item Elements for our response: The paper takes one national election, simulates many more observation points by adding random noise (noise that is plausible given observed district-level data), and then estimates partisan bias and components from simulated data. The approach is flexible and can be applied to different research designs. If conclusions over a longer period are of interest (to, say, investigate bias before/after an electoral reform, or to study a given "party system"), the analyst could pool elections in the period(s) and either use the Linzer multiplication approach (one election at a time, then pool simulation), or simply use the pooled data for direct estimation. The revised manuscript makes note of this in the concluding remarks.
\end{itemize}
\item {\bfseries\sffamily DONE} Very minor point: Population levels in districts are based on linear interpolations between censuses. Following standard models of population growth, it would be more appropriate to assume constant population growth rates over the time interval, which can easily be calculated (meaning that population growth in absolute numbers will be smaller for earlier years, if growth is positive).
\label{sec:orgheadline13}
\begin{itemize}
\item EM action: defend our approach, discuss alternative in on-line appendix.
\item Elements for our response: Estimating intercensal populations in units of analysis raises several challenges. The key problem appears to be the choice of a functional form that both smooths the rate of population growth while also taking the values actually observed on three census years (2000, 2005, and 2010). Applying an exponential form between pairs of census years does a good job for years between observations, but not for the time before and after, nor does it treat transitions from one pair to the next smoothly. A polynomial form would allow work with all three census counts, but is problematic for projecting estimates beyond 2010. Since these alternate estimation methods requires demographic knowledge beyond that which is available, we opted for the simpler linear estimation instead. We discuss the rationale for our linear estimation method, and the challenges of the non-linear approach, in the on-line appendix.
\end{itemize}
\item {\bfseries\sffamily NO\_ACTION\_NEEDED} All in all, I learned a lot from reading this paper, and I think it is a very good piece of scholarly work. I recommend that the authors are given the chance to revise and resubmit the paper.
\label{sec:orgheadline14}
\begin{itemize}
\item EM action: none.
\end{itemize}
\end{enumerate}

\section{{\bfseries\sffamily DONE} Reviewer \#3}
\label{sec:orgheadline19}
\begin{enumerate}
\item {\bfseries\sffamily NO\_ACTION\_NEEDED} This article was a pleasure to read and to evaluate. It is well written, with a theory clearly presented, interesting findings, and a contribution to the estimation of seats and votes in multi-party systems. The article brings together three different traditions in the study of seats and votes, unifying them in a model that allows researchers to discriminate the sources of biases in multi-party races. The article will be of interest to those that conduct basic research on seat-vote models as well as those interested in the mechanical properties of electoral rules in Mexico. Consequently, I recommend publication as is.
\label{sec:orgheadline16}
\begin{itemize}
\item EM action: none.
\end{itemize}
\item {\bfseries\sffamily NO\_ACTION\_NEEDED} I do not have recommended changes to the article. The article is honest in stating that it is an original and interesting improvement on existing models but not a radically different modeling strategy. I appreciate that the article does not try to oversell their contribution or findings.
\label{sec:orgheadline17}
\begin{itemize}
\item EM action: none.
\end{itemize}
\item {\bfseries\sffamily DONE} That said, the authors could be more aggressive in the introduction to convey to readers how the current article changes prior conventional wisdom in the Mexican election and what contributions will result from estimating their model in other electoral systems. Beyond that, I could only hope that every manuscript I have to review would be such an easy and interesting reading.
\label{sec:orgheadline18}
\begin{itemize}
\item EM action: conceded.
\item Elements for our answer: We now stress, in the introduction, how our procedure opens up the comparative study of electoral systems. We mention Canada, India, the UK, France, Australia, Chile, and Ireland as some of the cases for the comparative study of partisan bias. Given that we pursued Reviewer \#2's recommendation to downplay the Mexican case study in the framing, we have not stressed how our findings jibe with the conventional wisdom in the introduction---we leave this in the discussion that closes the manuscript.
\item For co-authors: Mike might mention something on his student's work on the UK.
\end{itemize}
\end{enumerate}
\section{{\bfseries\sffamily DONE} Reviewer \#4}
\label{sec:orgheadline30}
\begin{enumerate}
\item {\bfseries\sffamily NO\_ACTION\_NEEDED} This paper intends to identify the relative and (assumed independent) impact of three different components of partisan bias  in the Mexican electoral system. Application case are lower-chamber federal legislative elections 2003-2012. The paper focuses exclusively on the single member districts  component of the Mexican electoral system.
\label{sec:orgheadline20}
\begin{itemize}
\item EM action: none.
\end{itemize}
\item {\bfseries\sffamily DONE} The objective of this paper is ambitious. The authors claim to combine most important methodical contributions in this context (Grofman et al. 1997, King 1990, Linzer 2012).  More specifically, it aims at separating the relative impact of malapportionment, boundary delimitations, and differential turnout in an additive multinomial logit model. Sometimes, one has the impression, that a failed redistricting reform is at the center of the paper --- which is confusing.
\label{sec:orgheadline21}
\begin{itemize}
\item EM action: conceded.
\item Elements for our response: By de-emphasizing the Mexican case in order to highlight the methodological contribution, we have also done our best to address Reviewer 4's judgment that the failed redistricting reform is confusingly at the center of the paper. We now make clear that our analysis  uses the map that was not implemented as part of the hypothetical analysis, offering perspective on the effect of reducing malapportionment (much else constant) on our measure of partisan bias and its components.
\end{itemize}
\item {\bfseries\sffamily DONE} Description and discussion of the Mexican electoral system is negligent. The author writes: "Section 4 describes Mexico's mixed-member electoral system, isolating the plurality tier for analysis" (p. 5). The description of the electoral system can actually be found in footnote 5. But  what does 'isolating' mean? The authors state: "We examine, in isolation, the elections held in the single member plurality-win districts. We do so because all voting and most campaigning take place in the plurality tier." The reviewer considers this legitimation as not sufficient. Note that the current Mexican electoral system includes stipulations balancing excessive partisan bias and including compensation schemes. More specifically, the electoral law prescribes an upper bound of seat-vote deviation of 8\%. The is not referred in the paper. Why ? Actually, estimating partisan bias for the SMD component exclusively without referring to this stipulation seems to be misleading.
\label{sec:orgheadline22}
\begin{itemize}
\item EM action: conceded.
\item Elements for our response: While analysis of SMD seats without the compensatory PR tier can be defended, it is no longer necessary. We have focused the paper, as recommended, on the methodological contribution, and thus the application the plurality tier serves as an illustrative (and interesting) application on its own. We have nonetheless added a paragraph (in section 4) elaborating how the substantive partisan bias results presented for \textbf{plurality seats only} have implications for the larger mixed system. We also added the 8 percent over-representation rule, which we had previously neglected to mention.
\end{itemize}
\item {\bfseries\sffamily DONE} One would never seriously propose, to measure partisan bias exclusively for the SMD component in a mixed system like, e.g. the German electoral system.
\label{sec:orgheadline23}
\begin{itemize}
\item EM action: conceded.
\item Elements for our response: Answered above.
\end{itemize}
\item {\bfseries\sffamily DONE} It is reasonable to expect strategic coordination of parties, candidates and voters in this context. The paper does not propose a theory what partisan bias means in such a setting. The authors cite Calvo/Micozzi (2005) but do not systematically take into account their arguments, especially the insight that "with more than two parties the relative change in seats depends critically on changes in the number of parties" (Calvo/Micozzi p. 1051)
\label{sec:orgheadline24}
\begin{itemize}
\item EM action: defend our approach, mild concession.
\item Elements for our response:  If the vote threshold to win another seat can be anticipated, strategic coordination should result in an the attempt to pool votes (or remove opportunities to spread votes thin) in order to reach that threshold. Other things constant, the threshold should be lower for bias-favored parties than for other parties, and partisan bias should therefore remove incentives for bias-favored parties to coordinate strategically with others. If that were so, however, bias-unfavored parties have incentives to coordinate, joining forces in an attempt to overcome their disadvantage by accruing the large-party bonus associated with parameter rho. These simple statements suggest one obstacle to a theory of what partisan bias means under strategic coordination: it is unlikely that other things remain constant. In particular, gauging the relation between partisan bias and strategic coordination is contingent also on parameter rho. A formal exposition of this is Cox and Katz's (2002, chapter 3) model of the value to parties of redistricting plans, where utility is a function of both lambda and rho. Since our paper is focused in lambdas only, we do not undertake such a theory.
\item More elements for our response: Regarding multipartism, Calvo and Micozzi show that increasing party competition pulls the votes-seats curve leftward (fig. 1-c). This is precisely what partisan bias achieves---but for a favored party only: a more efficient votes to seats conversion. Bias produces a rightward shift for unfavored parties: a less efficient votes to seats conversion. So whereas the effect of multipartyism remains symmetric, partisan bias does not. We have added a footnote at the end of section 1 to cover this.
\item Questions for co-authors: Should we keep the new foonote, or drop this? Can it be clarified?
\end{itemize}
\item {\bfseries\sffamily DONE} Section 3 is titled 'Measurement via Monte Carlo simulation'. I guess, the authors mean 'estimation'. The one-page section is not very instructive --- it provides no detailed insights to the applied procedure. The description of the modeling approach is not precise enough. (Some details are provided in footnote 14 in the results section).  More details of the formal setup (including the electoral system) are necessary. The derivation from, and the combination of the existing approaches should be precise and transparent.
\label{sec:orgheadline25}
\begin{itemize}
\item EM action: conceded.
\item Elements for our response: Section 3 now describes the modeling approach more explicitly. We have also written an on-line appendix (included along this re-submission) to accompany the article when it is  published. The appendix elaborates on the full applied procedure in a step-by-step approach, also serving as introduction to the code to replicate the analysis (that will be posted on-line upon publication). In particular, the appendix offer detail about the Monte Carlo to generate a large number of hypothetical national elections for each year (the Linzer method), and how three methods combine into our proposed procedure.
\item More elements for our response: We also replaced `measurement' with `estimation' in the section name.
\item More elements for our response: The attached appendix is still a work in progress, we plan to polish it if the article gets accepted.
\end{itemize}
\item {\bfseries\sffamily DONE} The authors use a multinomial logit type of model --- I missed a discussion of the crucial assumption of the independence of irrelevant alternatives (IIA) which implies equal substitution patterns which may not be met.
\label{sec:orgheadline26}
\begin{itemize}
\item EM action: conceded, the point is now elaborated in the on-line appendix (with no mention in the text, it is too specific).
\item Element for our response: Our multinomial logistic regression type of model satisfies the independence of irrelevant alternatives assumption in the same way that King's model does. Quoting him (King p. 168): "the implied assumption of independence of irrelevant alternatives is satisfied here, since the entire stochastic component is conditional on all parties and votes. The only random choice being made is by the electoral system in assigning seats to parties. Therefore, I use the multinomial probability distribution for the number of seats allocated to the J political parties, a straightforward generalization of the binomial". The only difference is our use of P binomial distributions instead of the multinomial.
\end{itemize}
\item {\bfseries\sffamily DONE} How do the authors account for districts with varying sets and sizes of candidates in the estimation?
\label{sec:orgheadline27}
\begin{itemize}
\item EM suggested action: clarified in text and expanded in the on-line appendix.
\item Elements for our response: Districts with varying sets and sizes of candidates pose an obstacle to the Linzer simulation and, in multi-year research designs, to the MCMC estimation. Linzer's (p. 405) approach overcomes this obstacle by treating subsets of districts with different patterns of party contestation separately in the mixture model. The simulated national elections include all parties that contested one district at least (and were not dropped by the analyst at the start of the process). Our single-year research design avoids the obstacle in MCMC estimation (the analyst can adapt the Bugs model to the number of parties in the simulated elections). Our code (see Table A2), however, is prepared to tackle a multi-year problem: a set of dummy variables, one for each party in the analysis, equal 1 if the party contested the election and 0 otherwise, is computed from the data at the outset and fed to the MCMC process; each numerator and denominator additive components (the party's lambda * v\(^{\text{rho}}\)) is multiplied by the corresponding dummy, so that parties not contesting drop from the likelihood function. We have expanded this in the appendix, with a mention in text.
\end{itemize}
\item {\bfseries\sffamily DONE} On p. 21, the authors state: "Leaving aside the question of how meaningful the estimated quantities are\ldots{}"  I admit that this statement is somewhat disturbing. The interpretation and usage of the estimated effects seem to be problematic: it is meanwhile established knowledge that coefficients of multiple nonlinear functions (as in MNL)  cannot be interpreted simply based on statistical significance, and even on the sign of a coefficient. Covariates have to be explicitly fixed for explicit values in order to get conditional probabilities / market shares, marginals and elasticies. Insafar the study should reassess the impact of the estimated coefficients for relevant and typical situations, and for the respective partisan biases in terms of  changes in market shares. E.g., illustrative scenarios in Table 1 could be provided for estimated coefficients.
\label{sec:orgheadline28}
\begin{itemize}
\item EM action: text explains in section 6 why we still discuss individual lambda coefs first and then assess impact through swing ratios.
\item Elements for our response: In the revised text, we have dropped the claim that "Leaving aside the question of how meaningful the estimated quantities are\ldots{}" that seems to have triggered Reviewer \#4's concern. Yet the general problem remains, and no easy solution seems to be in our sight. We are aware that, unlike OLS coefficients, the logit link in our model is an obstacle for the assessment of individual lambdas' impact of the DV. One common approach (e.g., clarify) is comparative statics analysis, letting one regressor of interest fluctuate while all others remain constant at mean, mode, or other illustrative values. This approach is inapplicable to partisan bias in a multi-party setting, due to the compositional nature of vote shares (the regressors): when v\(_{\text{p}}\) fluctuates, all other vote shares do not remain constant. "Proportional swing" models (cites) remove this complication by assuming that votes are won/lost relative to other parties' sizes. Instead of relying on such restrictive approach, the revised manuscript proceeds like the original submission did: discussing lambda estimates' magnitude and polarity first, then assessing their importance through swing ratios analysis of simulated elections---like Linzer does. We have added a footnote towards the end of section 6 providing a complete rationale for  how we proceed.
\end{itemize}
\item {\bfseries\sffamily DONE} Table 3 uses OLS regression for deriving swing ratios: "We derive swing ratios by regressing a party's seat shares in simulated elections on the party's simulated vote shares." Does this regression account for the uncertainity/credibility of simulated shares?
\label{sec:orgheadline29}
\begin{itemize}
\item EM action: explain that, in fact, it does.
\item Elements for our response: Linzer (p. 408) suggests using OLS regressions as an alternative for deriving swing ratios ("Although equation (4) requires no parametric assumptions about the functional relationship between [party p's vote share and the p's expected simulated seat share], the relationship between simulated seat shares \ldots{} and simulated vote shares \ldots{} around [p's mean vote share] will usually be approximately linear. In that event, the slope of a linear regression of [p's simulated seat shares] on [p's simulated vote shares] will be roughly equivalent to the swing ratio estimate"). Linzer simulations represent the plausibility of various national-level election outcomes---both vote shares and seat shares---given the observed district-level conditions of that election. The uncertainty (a, say, 95-percent confidence interval) of the swing ratio estimate is obtained sorting simulated seat shares for a given vote share (typically the mean simulated vote share), and taking the 2.5 and 97.5 percentiles. Our regressions report coefficient standard errors instead: since standard errors are derived from the very same simulations, they account for  the same sources of uncertainty as the confidence interval. We could alternatively report 95-percent CIs around predicted seat shares: these would require more explanation than standard SEs, so we prefer the latter. We have added a section in the on-line appendix elaborating this point.
\item Co-authors: does this make sense?
\end{itemize}
\end{enumerate}
\section{ToDo list}
\label{sec:orgheadline41}
\begin{enumerate}
\item Write cover letter explaining changes. The above list of issues has all the substance needed for this letter (and me may even choose to just send that list mostly as it is!). Letter should mention that we re-did all analysis to include 2015 election returns (previously unavailable), and also adding back secciones that were split in the period due to overpopulation. These had been dropped to save time. These units are relatively unimportant in sheer numbers (175 overpopulated secciones were split into 5034 new units in the period). But they are concentrated in suburban areas with fast demographic growth since the 1990s. Estimates for 2003--2012 have changed, but they tell the same general story.
\label{sec:orgheadline31}
\item Conclusion needs to be adapted to the methodological framing---present version seems to emphasize too much the substantive findings.
\label{sec:orgheadline32}
\item Micah/Mike: Which repository for data, code, appendix? github? ericmagar.com? dataverse? several?
\label{sec:orgheadline33}
\item {\bfseries\sffamily DONE} Re-do rri plots with cleaner seccion-to-dostroct aggregations for paper
\label{sec:orgheadline34}
\item {\bfseries\sffamily DONE} Re-do bias estimate plots with 2015 in for paper
\label{sec:orgheadline35}
\item {\bfseries\sffamily DROPPED} Decide if we call it the 2013 map or the 2015 map.
\label{sec:orgheadline36}
\item Make sure census gap mentioned in the text: I mention it in the appendix without introduction
\label{sec:orgheadline37}
\item Mike: The two comments I received from MPSA were:
\label{sec:orgheadline40}
\begin{enumerate}
\item {\bfseries\sffamily DONE} Need a little more detail on the MCMC algorithm
\label{sec:orgheadline38}
\item Need more context for non-Mexico scholars
\label{sec:orgheadline39}
\end{enumerate}
\end{enumerate}

\section{ToDo list if we get publication}
\label{sec:orgheadline46}
\begin{enumerate}
\item Remove circularities btw red.r and analizaEscenarios.r
\label{sec:orgheadline42}
\item {\bfseries\sffamily DONE} verify that error in king's denominator in red.r is innocuous
\label{sec:orgheadline43}
\item Drop above from spaghetti code (never used for Linzer estimation)
\label{sec:orgheadline44}

\item Turn various code files (red.r, linzerElas.r, analizaEscenarios.r\ldots{}) into single---if longer---script
\label{sec:orgheadline45}
\end{enumerate}
\end{document}
